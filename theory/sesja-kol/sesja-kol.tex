\documentclass[12pt]{beamer}
\mode<presentation>
\definecolor{pjwgreen}{rgb}{0.196,0.349,0.157}
\usetheme{Frankfurt}
\setbeamercolor*{palette primary}{fg=white,bg=pjwgreen}
\setbeamercolor*{palette secondary}{fg=white,bg=pjwgreen}
\setbeamercolor*{palette tertiary}{fg=white,bg=pjwgreen}
\setbeamercolor*{palette quaternary}{fg=white,bg=pjwgreen}
\setbeamercolor{structure}{fg=pjwgreen}\setbeamercolor{local
structure}{parent=structure}

\usepackage[utf8]{inputenc}
\usepackage[polish]{babel}
\usepackage{polski}
\usepackage{amsmath}
\usepackage{amsfonts}
\usepackage{amssymb}
\usepackage{graphicx}
\usepackage{wrapfig}
\usefonttheme{professionalfonts}
\usepackage{pgfplots}

\newcommand\Fontvi{\fontsize{14pt}{7.2}\selectfont}

\author{Paweł J. Wal\\Opiekun pracy: dr inż. Łukasz Rauch}
\title[Implementacja solwera frontalnego...]{Implementacja solwera frontalnego zrównoleglonego w wielowęzłowym heterogenicznym środowisku sprzętowym}
\institute[AGH]{
  Wydział Inżynierii Metali i Informatyki Przemysłowej\\
  Akademia Górniczo-Hutnicza im. Stanisława Staszica w Krakowie
}

\setbeamercovered{transparent} 
\setbeamertemplate{navigation symbols}{} 
\logo{} 
\date{7 maja 2015} 
%\subject{} 
\begin{document}

\begin{frame}
\titlepage
\hspace{80pt}
\includegraphics[scale=0.20]{logo_ms}
\end{frame}

\section{Motywacja pracy}
\subsection{Oryginalny pomysł - solwer multifrontalny na GPU}

\begin{frame}{Oryginalny pomysł - solwer multifrontalny na GPU}
\begin{itemize}
	\item Wykorzystany OpenCL
	\item Masowo równoległe przetwarzanie na GPU
\end{itemize}
\vspace{10pt}
\begin{itemize}
	\item Wydzielenie frontów rozwiązania
	\begin{itemize}
	    \item Podział macierzy na części (dekompozycja)
	    \item Wydzielenie z macierzy rzadkiej gęstych fragmentów
	\end{itemize}
	\item Przetworzenie frontów rozwiązania
	\begin{itemize}
    	\item Przetwarzanie części niezależnie od siebie
	    \item Rozwiązanie zależności - na koniec
	\end{itemize}
\end{itemize}
\end{frame}

\subsection{Oryginalny pomysł - dekompozycja}

\begin{frame}{Oryginalny pomysł - dekompozycja}
\begin{center}
	\includegraphics[scale=0.4]{dekompozycja}
\end{center}
\end{frame}

\subsection{Wady oryginalnego podejścia}
\begin{frame}{Wady oryginalnego podejścia}
\begin{itemize}
	\item Użycie tylko jednego z dostępnych urządzeń
	\item Uwarunkowania pamięciowe
	\begin{itemize}
	    \item Ładowanie stosunkowo dużych fragmentów macierzy do pamięci
	    \item Dodatkowa pamięć potrzebna na specyficzne struktury danych używane przez solwer (mapowanie wierszy)
	\end{itemize}
	\item Niewielka odporność na zdarzenia katastrofalne (np. awarię zasilania)
	
\end{itemize}
\end{frame}

\subsection{Nowy kierunek}
\begin{frame}{Nowy kierunek}
\begin{itemize}
	\item Użycie wszystkich dostępnych urządzeń
	\item Wykorzystanie możliwości maszyn z wieloma GPU?
	\begin{itemize}
	    \item Wiele GPU = specjalne płyty główne i obudowy = koszt
	    \item Wiele GPU = specjalne systemy operacyjne i konfiguracje = maszyna o wąskim zastosowaniu
	    \item Potęga GPU = moc urządzeń klasy konsumenckiej
	\end{itemize}
	\item Myśl przewodnia: realne zastosowanie w przemyśle i nauce	
\end{itemize}
\end{frame}

\section{Koncepcja}
\subsection{Założenia projektu}
\begin{frame}{Założenia projektu}
\begin{itemize}
	\item Możliwość wykorzystania dziesiątek, setek, ... urządzeń dowolnej klasy - przez OpenCL
	\item Dowolna lokalizacja urządzeń
	\item Dowolne dopuszczalne obciążenie urządzeń
	\item Brak nierealistycznych lub wygórowanych założeń (najlepiej: brak założeń w ogóle):
	\begin{itemize}
		\item Ilość urządzeń w maszynie
		\item Moc obliczeniowa urządzeń
		\item Rodzaj i jakość połączenia (filtrowane, komórkowe)
		\item Czas i wydajność pracy	
	\end{itemize}
	\item Wiele problemów na raz, niezależnie od wielkości
	\item Odporność na błędy
\end{itemize}
\end{frame}

\subsection{Inspiracja}
\begin{frame}{Inspiracja}
\begin{center}
\includegraphics[scale=0.30]{sah_logo}\\ \vspace{25pt}
\includegraphics[scale=0.20]{boinc_logo} \\ \vspace{25pt}
\includegraphics[scale=0.40]{folding_logo}
\end{center}
\end{frame}

\subsection{Schemta rozwiązania}
\begin{frame}{Schemat rozwiązania}
\begin{center}
\includegraphics[scale=0.60]{schemta}\\ \vspace{25pt}
\end{center}
\end{frame}

\subsection{Przewaga rozwiązania}
\begin{frame}{Przewaga rozwiązania}
\begin{itemize}
\item Wykorzystanie ukrytego (marnowanego) potencjału urządzeń
\item Niewielkie wymagania własne 
\begin{itemize}
	\item Pracuje z dostępnymi technologiami i popularnymi urządzeniami
	\item Nie wymaga inwestycji
	\item Nie wymaga specjalistycznej wiedzy by być skutecznie użytkowanym
\end{itemize}
\item Pozwala na rozwiązywanie problemów niepraktycznych lub niemożliwych do rozwiązania
\end{itemize}
\end{frame}

\section{Realizacja}
\subsection{Architektura}
\begin{frame}{Architektura}
\begin{itemize}
	\item Klient-serwer
	\item Dlaczego nie MPI?
	\begin{itemize}
		\item Rozłączne LANy
		\item Pojedyncze maszyny w różnych lokalizacjach
		\item Gorsze połączenia
	\end{itemize}
	\item Ostateczna decyzja? Protokół HTTP!
	\begin{itemize}
		\item Popularność
		\item Typowość - raczej nie jest blokowany
	\end{itemize}
\end{itemize}
\end{frame}

\subsection{Oprogramowanie}
\begin{frame}{Oprogramowanie}
\begin{itemize}
	\item Serwer: Ruby on Rails
	\begin{itemize}
		\item Jednolite, standardowe API (REST)
		\item Bezstanowość
		\item Ta sama usługa dostarcza punkt wejścia dla klientów i dla użytkowników (interfejs graficzny)
	\end{itemize}
	\item Klient: Ruby
	\begin{itemize}
		\item Dynamiczny rozwój oprogramowania
	\end{itemize}
	\item Elementy obliczeniowe: C/C++
\end{itemize}
\end{frame}

\subsection{Przede wszystkim łatwość}
\begin{frame}{Przede wszystkim łatwość}
\begin{itemize}
	\item Łatwość rozszerzania serwera - prosty język,\\ jasna struktura danych
	\item Łatwość modyfikacji klienta
	\item Łatwość stworzenia zupełnie nowego klienta
	\item Łatwość dostosowania do aktualnie postawionego celu przez specjalistę
	\item Łatwość użycia jako gotowego rozwiązania przez specjalistę w innej dziedzinie
\end{itemize}
\end{frame}

\subsection{Serwer - interfejs użytkownika}
\begin{frame}{Serwer - interfejs użytkownika}
\begin{center}
	\includegraphics[scale=0.3]{swarmhost}
\end{center}
\end{frame}

\section{Wyniki pracy}

\subsection{Wyniki dotychczasowych badań}
\begin{frame}{Wyniki dotychczasowych badań}
\begin{center}
	\includegraphics[scale=0.45]{proof}
	\newline
	\tiny Jack: 4x nVidia Tesla M2090, 1x Intel Xeon X5650. Jennah: 1x nVidia GeForce GT630.\\Serwer: 1x Intel Core i7-4710HQ, 16 GB RAM.
\end{center}
\end{frame}


\subsection{Dalszy rozwój}
\begin{frame}{Dalszy rozwój}
\begin{itemize}
	\item Rozwiązanie jest nadal w rozwoju
	\item Aktualnie badane obszary:
	\begin{itemize}
		\item Usprawnienie komunikacji po HTTP
		\item Obniżenie obciążenia serwera
		\item Zwiększenie asynchroniczności kluczowych elementów
		\item Badanie wpływu parametrów podziału macierzy na prędkość uzyskiwania wyników
		\item Badanie wpływu parametrów uruchomieniowych OpenCL na szybkość uzyskiwania rozwiązań na kliencie
		\item I inne.
	\end{itemize}
\end{itemize}
\end{frame}


\end{document}